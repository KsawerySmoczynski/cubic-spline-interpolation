\section{Interpolujące funkcje sklejane stopnia trzeciego (sześcienne)} 
\label{sekcja1:wstęp}
Poniższa praca będzie miała na celu przybliżenie zagadnienia interpolujących funkcji 
sklejanych stopnia trzeciego (sześciennych). W drugiej sekcji - \nameref{sekcja2:definicja} 
wprowadzona zostanie formalna definicja ww. funkcji wraz z ich postacią macierzową 
(subsekcja \nameref{subsekcja2:postac_macierzowa}) oraz twierdzenie z którego wynika, że
jest są to najgładsze funkcje interpolujące (subsekcja \nameref{subsekcja2:gladkosc}).
Następnie w sekcji ...
