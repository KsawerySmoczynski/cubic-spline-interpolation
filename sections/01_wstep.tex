\section{Interpolujące funkcje sklejane stopnia k} 
\label{sekcja1:wstęp}
Poniższa praca będzie miała na celu przybliżenie zagadnienia interpolujących funkcji 
sklejanych stopnia k. W drugiej sekcji - 
\nameref{sekcja2:definicja} 
wprowadzona zostanie formalna definicja ww. funkcji wraz z jej postacią macierzową 
(subsekcja \nameref{sekcja2.1:postac_macierzowa}) oraz twierdzenie z którego wynika, że
jest to najgładsza funkcja interpolująca (subsekcja \nameref{sekcja2.2:gladkosc})
