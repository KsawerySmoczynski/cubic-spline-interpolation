\section{Definicja}
\label{sekcja2:definicja}
W pierwszej kolejności zdefiniujmy funkcje sklejane stopnia k, jako
 generalizację funkcji sklejanych stopnia trzeciego. 
 Następnie uszczegółowimy konstrukcję funkcji sklejanych stopnia
  trzeciego i opiszemy ich własności. 
  W następnej kolejności przedstawimy postać macierzową 
  ww. funkcji by ostatecznie przedstawić twierdzenie z 
  którego wnioskujemy o tym, że są to najgładsze funkcje interpolujące.

\subsection{Definicja funkcji sklejanych stopnia $k$}
\label{subsekcja2:definicja}
Zdefniujmy $\Pi_k$ jako wielomian stopnia $k$, gdzie $k \in \mathbb{Z_+}$.
Niech będzie dana funkcja $S$ określona na pewnym 
przedziale $(a,b)\in\mathbb{R}$, ustalmy $n+1$ węzłów $T = \{t_i \in (a,b), i \in \{0, 1, 2, \dots, n\}: t_1 < ... < t_{i-1} < t_{i} < t_{i+1} < t_{n+1}\}$.
Wówczas dla każdej liczby $k \in \mathbb{Z_+}$ funkcją sklejaną 
stopnia $k$ nazywamy taką funkcję $S$, która spełnia 
następujące warunki:
\begin{enumerate}
    \item Dla każdego przedziału $[t_i, t_{i+1})$, gdzie $i \in \{0, 1, 2, \dots, n-1\}$ funkcja $S$ jest wielomianem $\Pi_k$ stopnia $k$.
    \item Funkcja $S$ na przedziale $[t_0, t_n]$ ma ciągłą $(k-1)$-szą pochodną.
\end{enumerate}

\subsection{Opis i własności funkcji sklejanych stopnia trzeciego (sześciennych)}
\label{subsekcja2:opis_i_wlasnosci_fss3}
Niech dane będą węzły $t_i$ oraz odpowiadające im 
wartości $y_i$, gdzie $i \in \{0, 1, 2, \dots, n\}$. 
Zdefiniujmy, funkcję $S$ która dla każdego węzła $t_i$ przyjmuje 
wartość $y_i$ \big($\forall_{i\in\{0,1,2,\dots,n\}} \space S(t_i) = y_i$\big)
oraz która dla każdego z przedziałów $[t_i, t_{i+1})$ jest wielomianem $S_i$ stopnia trzeciego $\Pi_3$.
Aby funkcja S była funkcją sklejaną trzeciego stopnia musi spełniać warunki zdefiniowane w poprzedniej subsekcji (\nameref{subsekcja2:definicja}).


\subsection{Postać macierzowa}
\label{subsekcja2:postac_macierzowa}
\subsection{Gładkość sześciennych funkcji sklejanych}
\label{subsekcja2:gladkosc}
